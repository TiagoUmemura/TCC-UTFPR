\documentclass[12pt,english,brazil,a4paper,utf8,oneside]{utfpr-tcc}

% Este comando não é necessário: utilizei apenas para deixar o latex2rtf
% feliz (e descobrir a codificação do texto).
\usepackage[utf8]{inputenc}

% Suporte a figuras e subfiguras
\usepackage{graphics}
\usepackage{subfigure}

% Suporte a tabelas (principalmente do cronograma)
\usepackage{tabularx}
\usepackage{multirow}
\usepackage{array}
\usepackage{tabularx}
\usepackage{colortbl}
\usepackage{hhline}
\usepackage{xcolor}

% Elementos geralmente utilizados na tabela do cronograma
\newcommand{\fullcell}{\multicolumn{1}{>{\columncolor[gray]{0.5}}c}{}}
\newcommand{\fullcellline}{\multicolumn{1}{>{\columncolor[gray]{0.5}}c|}{}}
\newcommand{\mc}[3]{\multicolumn{#1}{#2}{#3}}
\newcommand{\y}{\rule{8pt}{4pt}}
\newcommand{\n}{\hspace*{8pt}} 

% Define o caminho das figuras
\graphicspath{{images/}}

% Dados do curso que não precisam de alteração
\university{Universidade Tecnológica Federal do Paraná}
\universityen{Federal University of Technology -- Paraná}
\universityunit{Departamento Acadêmico de Computação}
\address{Campo Mourão}
\addressen{Campo Mourão, PR, Brazil}
\documenttype{Monografia}
\documenttypeen{Monograph}
\degreetype{Graduação}


%%%%%%%%%%%%%%%%%%%%%%%%%%%%%%%%%%%%%%%%%%%%%%%%%%%%%%%%%%%%%%%%%%%%%%%%%%%%%
% Alterar daqui para baixo
%%%%%%%%%%%%%%%%%%%%%%%%%%%%%%%%%%%%%%%%%%%%%%%%%%%%%%%%%%%%%%%%%%%%%%%%%%%%%

% Dados do curso. Caso seja BCC:
\program{Curso de Bacharelado em Ciência da Computação}
\programen{Undergradute Program in Computer Science}
\degree{Bacharel}
\degreearea{Ciência da Computação}
% Caso seja TSI:
% \program{Curso Superior de Tecnologia em Sistemas para Internet}
% \programen{Undergradute Program in Tecnology for Internet Systems}
% \degree{Tecnólogo}
% \degreearea{Tecnologia em Sistemas para Internet}


% Dados da disciplina. Escolha uma das opções e a descomente:
% TCC1:
\goal{Proposta de Trabalho de Conclusão de Curso de Graduação}
\course{Trabalho de Conclusão de Curso 1}
% TCC2:
% \goal{Trabalho de Conclusão de Curso de graduação}
% \course{Trabalho de Conclusão de Curso 2}


% Dados do TCC (precisa alterar)
\author{Tiago Kenji Umemura}  % Seu nome
\title{Impacto da Entropia em fatores sociais} % Título do trabalho
\titleen{} % Título traduzido para inglês
\advisor{Prof. Dr. Igor Scaliante Wiese} % Nome do orientador. Lembre-se de prefixar com "Prof. Dr.", "Profª. Drª.", "Prof. Me." ou "Profª. Me."}
% \coadvisor{} % Nome do coorientador, caso exista. Caso não exista, comente a linha.
\depositshortdate{2016} % Ano em que depositou este documento

% Dados da ficha catalografica. Ela é opcional, mas é uma boa ideia inserí-la. Exemplos para geração (http://fichacatalografica.sibi.ufrj.br/)
\fichacatautor{}  % Nome conforme citado (ou seja, no formato "Sobrenome, Nome").
\fichacatbib{Biblioteca da UTFPR de Campo Mourão} % Não alterar
\fichacatpum{M488} % Código Cutter-Sanborn. Use a primeira letra do sobrenome seguido do número conforme as primeiras letras do sobrenome e a tabela http://www.amormino.com.br/cutter-sanborn/cutter1.html
\fichacatpalcha{} % Assuntos do trabalho. Cada item deve ser enumerado e separado por ponto: 1. xxx. 2. yyy. 3. zzz.
\fichacatpdois{} % Deixar em branco


\begin{document}
	
\frontmatter
\maketitle

\begin{resumo}

% TODO: se possível, escreva um resumo estruturado. Para TCC 1, o resumo estruturado teria os seguintes elementos:
A entropia de mudança calcula o quanto um software sofre alterações em um determinado período de tempo e está relacionada com o número de defeitos de um software, porém, além da entropia é importante observar os fatores sociais que podem ser bons indicadores de qualidade de software. Nesse trabalho será analisado a relação da entropia com os fatores sociais e qualidade de software.

\textbf{Contexto:} A pesquisa será realizada utilizando os repositórios do sistema de controle de versões Git.\\
 \textbf{Objetivo:} Identificar quais fatores sociais tem maior impacto na entropia e na qualidade do software. \\
 \textbf{Método:} A entropia e os fatores sociais serão calculados utilizando dados extraídos da base de dados GHTorrent, API Github e a ferramenta change-metrics.\\
 \textbf{Resultados esperados:} É esperado que maior entropia esteja relacionada com o débito social.
% ou, para TCC 2:
% \textbf{Contexto:} \\
% \textbf{Objetivo:} \\
% \textbf{Método:} \\
% \textbf{Resultados:} \\
% \textbf{Conclusões:}

% Palavras-chaves, separadas por ponto (tente não definir mais do que cinco)
\palavraschaves{Entropia. Fatores Sociais. Qualidade. Defeitos.}
\end{resumo}

%quais fatores tem incluencia na entropia (main quest)
%visualização de indicadores dos fatores com a entropia (side quest)

% Caso seja TCC 2, precisa traduzir o resumo e as palavras-chaves para inglês:
% \begin{abstract}
% \textbf{Context:}
% \textbf{Objective:}
% \textbf{Method:}
% \textbf{Results:}
% \textbf{Conclusions:}

% Palavras-chaves em inglês, separadas por ponto.
% \keywords{}
% \end{abstract}



% Listas (opcionais, mas recomenda-se a partir de 5 elementos)
\listoffigures
\listoftables

% Sumário
\tableofcontents

\mainmatter
% TODO: incluir arquivos latex com os capítulos
 \chapter{Introdução}

Os artefatos de softwares são modificados ao longo do tempo e nesse processo a qualidade do software tende a piorar \cite{Hassan:2009:PFU:1555001.1555024}, uma vez que existe a necessidade da mudança contínua de requisitos durante a evolução do software.

Para quantificar o impacto das mudanças contínuas os pesquisadores \cite{Hassan:2009:PFU:1555001.1555024}  tem utilizado o conceito de entropia de mudança. Esta medida pode ser obtida a partir do número de mudanças que ocorrem em um projeto ou arquivo em um determinado período de tempo. \citeonline{Hassan:2009:PFU:1555001.1555024} observou que maior quantidade de mudanças está relacionada com o aumento do valor da entropia e consequentemente está relacionado com maior tendência do software apresentar defeitos. 

O impacto da entropia na qualidade do projeto também é investigado na pesquisa de \citeonline{Canfora2014}, que analisou a relação entre a entropia e atividades de desenvolvimento, como a refatoração, padrões de projetos e a quantidade de desenvolvedores que mudam um determinado arquivo.

Estes estudos não consideram uma grande quantidade de métricas de software, por exemplo, não foi analisado a comunicação dos desenvolvedores, número de \textit{commits} e número de arquivos que são alterados. Portanto não existe uma ferramenta que possibilite os desenvolvedores e os gerentes monitorarem os efeitos da entropia e sua relação com métricas sociais, de autoria e de processo.

Diante desse contexto, sabe-se que o desenvolvimento de software é uma tarefa sócio-técnica, pois é essencial a comunicação entre os desenvolvedores para coordenar as atividades de desenvolvimento durante a evolução do software. Assim o objetivo deste trabalho é construir uma ferramenta que monitora os valores de entropia e a sua relação com as métricas sociais, de autoria de mudança (authorship, ownership e experience), métricas de processo (quantidade de commits, quantidade de defeitos, quantidade de linhas removidas, quantidade de linhas adicionadas, Code Churn, quantidade de refatorações, max change set e average change set) e número de defeitos.

A ferramenta irá gerar relatórios estatísticos sobre a entropia e as métricas e fornecerá a visualização desses dados para auxiliar os desenvolvedores na tomada de decisões durante o desenvolvimento de um projeto. O relatório estatístico será feito utilizando as técnicas de \textit{Wilcoxon}, \textit{ANOVA} e \textit{Cliffs's test} e irá sugerir quais arquivos do projeto possuem maior tendência para apresentar defeitos. Na visualização de dados será utilizado o \textit{Treemapping} para  mostrar quais arquivos do projeto possuem maior entropia e para as métricas de software serão utilizados gŕaficos que mostrem os valores das métricas ao longo do tempo.

A última etapa será a avaliação da ferramenta que será feita com desenvolvedores que utilizarão a ferramenta e depois responderão um questionário sobre as funcionalidades e usabilidadade da ferramenta.  

Esta proposta está organizada da seguinte forma. O capítulo 2 apresenta as definições das métricas sociais, de processos e os trabalhos relacionados. O capítulo 3 apresenta a proposta de construção da ferramenta, como a ferramenta será validada e uso da ferramenta. O capítulo 4 apresenta o cronograma.
 \chapter{Referencial Teorócio}
Este capítulo apresenta os conceitos de entropia, authorship, ownership, experience e qualidade de software que foram estudados em outras pesquisas.

\section{Entropia de mudança}
A entropia de \citeonline{Shannon:2001:MTC:584091.584093} é uma medida para mensurar a incerteza associada a uma variável que quantifica uma informação contida em uma mensagem produzida por um emissor de dados. É utilizada para determinar a quantidade de bits necessários para identificar unicamente um distribuidor de dados, assim quanto maior a entropia maior a incerteza para identifica-lo. A partir disso foi criado a entropia de mudança com o objetivo de calcular o quanto um código está mudando durante um determinado período de tempo. 

A entropia de mudança introduzida por \citeonline{Hassan:2009:PFU:1555001.1555024} considera que o software é o emissor de dados e as modificações realizadas são os dados a serem considerados. É uma medida para mensurar a quantidade de mudanças que ocorreram em um determinado espaço de tempo em um arquivo de um projeto, as mudança consideradas podem ser calculadas de acordo com a quantidade de linhas afetadas em um intervalo de tempo ou número de commits. Hassan também mostrou que alta entropia está relacionada a maior tendência do projeto apresentar falhas.

A pesquisa de \citeonline{Canfora2014} relaciona a entropia de mudança com características do software e atividades de desenvolvimento. As características analisadas foram: refatoração, número de commiters, padrões de projetos e nome de tópicos no projeto. Na pesquisa foram analisados projetos nos sistemas ArgoUML, Eclipse-JDT, Mozilla e Samba em um período de cerca de 10 anos e foi mostrado como esses fatores se relacionam com a entropia, que por consequecia impacta a qualidade do software. Os resultados de Canfora indicam que a entropia de mudança varia conforme o número de desenvolvedores aumentam, sendo que diferentes sistemas de controle de versão apresentam variações diferentes na entropia.


\section{Authorship e Ownership}
Authorship é uma medida para mensurar o quanto um desenvolvedor contribuiu para um determinado módulo de software e o ownership é o autor com maior authorship, assim um módulo de software pode apresentar ownership forte ou distribuído entre os contribuidores.

Essas duas medidas podem ser obtidas de várias formas: contando número de arquivos que o desenvolvedor modificou, número de commits e outra possibilidade é contar número de linhas modificadas pelo contribuidor, também chamada de code churn\cite{Munson:1998:CCM:850947.853326}.

No trabalho de Greiler e Kim Herzig\cite{Greiler} é feito um estudo para relacionar Ownership com a qualidade do código. Para medir a qualidade do código é considerado o número de bugs que foram corrigidos, número de diretórios e arquivos defeituosos e o Ownership é medido considerando número de contribuidores de um arquivo e também é verificado se existe um contribuidor principal, nesse caso a medida foi calculada com base no número de commits do autor em relação ao total de commits para aquele componente. Também foi mostrado que módulos com ownership fraco tendem a apresentar mais defeitos. Nesse mesmo trabalho foi estudado quatro diferentes software para Windows e é possível observar que nos sistemas analisados a métrica de ownership está correlacionado com número de defeitos. Isso é observado na análise feita tanto em nível de arquivo quanto em nível de diretório.  
Essa pesquisa mostrou que o número de contribuidores e a porcentagem de mudanças aplicadas pelo ownership são bons indicadores de qualidade , mostrando que há relação entre ownership e a qualidade de um projeto apesar da quantidade de erros também depender do tamanho do projeto.

Na pesquisa realizada por \citeonline{Rahman2011} é analisado a relação entre o número de autores com o ownership e com a qualidade do código. O authorship é calculado utilizando o número de linhas modificadas no código pelo desenvolvedor dividido pelo número total de linhas do arquivo, e o autor com a maior contribuição é denomidado ownership. Também é definido implicated code, que é o código modificado quando é corrigido um determinado erro no módulo de software. O trabalho de Rahman investiga a relação entre ownership, authorship e experience com implicated code. Para cada linha de código modificado é utilizado o comando blame para identificar o autor responsável por essa mudança. O resultado fornece evidências indicando que implicated code tende a ser mais frequentemente gerado por poucos autores, vários fragmentos de códigos modificados tem apenas um único autor.

O artigo de \citeonline{Foucault2015} também estuda como o ownership impacta na qualidade do código. Nele os contribuidores são classificados como owner, minor e major. Owner é o contribuidor com maior valor de contribuição, minor o desenvolvedor que contribuiu com menos de 5\% e major contribuiu com mais de 5\%. Após medir as métricas é comparado o número de erros com o a medida de ownership.

Além disso também há pesquisas\cite{Thongtanunam} que analisaram a diferença entre a contribuição de code authoring e reviewing e como a atividade de code reviewing afetam projetos com vários autores que contribuem pouco o que ajuda a compreender melhor o impacto do ownership na qualidade do código.

\section{Experience}
A experience do desenvolvedor influência na produtividade do mesmo e na qualidade do código produzido como foi mostrado em outras pesquisas\cite{Rahman2011}\cite{10.2307/2634607} e quanto mais o desenvolvedor trabalha em diferentes componentes do sistema maior será a sua experience.

Já foi feito pesquisas\cite{Rahman2011} mostrando a relação entre experience e qualidade do código além disso a experience é dividida em dois tipos: a experience especializada e experience geral. A experience especializada é medida considerando o quanto um indivíduo contribui em um determinado arquivo e a experiência geral é medida conseiderando um projeto inteiro.

A experience é a medida para calcular o nível de experiência do contribuidor, essa medida é computada analisando o número de linhas\cite{Rahman2011} deltas comitadas pelo contribuidor em determinado espaço de tempo. Além disso Rahman já mostrou que experiência especializada leva a produzir códigos com menor número de erros e ainda não foi possível concluir que a experiência geral contribui para aumentar a qualidade. No trabalho de Rahman foi correlacionado implicated code com experience.


% \chapter{Método}
Este capítulo apresenta as questões de pesquisas e o método para extrair e analisar as métricas de projetos que serão analisados.

QP1:Maior entropia tem como consequência débito social?

Abordagem: Utilizando o GHTorrent será extraído o número de commits de um projeto em um determinado período, em seguida será calculado a entropia desse projeto nesse mesmo período.

As informações extraídas do GHTorrent não são o suficientes para calcular as métricas sociais então é necessário utilizar também API do github para coletar outros dados.

Resultados esperados: \\

QP2:Maior entropia degrada a qualidade do projeto?

Abordagem: como será o procedimento para responder a QP2.

Resultados esperados: 
% \chapter{Cronograma}
\begin{table}[]
\centering
\caption{Tabela do cronograma}
\label{cronograma}
\begin{tabular}{|l|l|l|l|l|l|}
\hline
Módulos Ferramenta                     & Jan & Fev & Mar & Abr & Jun \\ \hline
Coletar dados                          & X   &     &     &     &     \\ \hline
Calcular entropia                      &     & X   &     &     &     \\ \hline
Calcular métricas                      &     & X   &     &     &     \\ \hline
Gerar Relatório de Análise Estatística &     &     & X   &     &     \\ \hline
Visualizar Dados                       &     &     &     & X   &     \\ \hline
Avaliar Ferramenta                     &     &     &     &     & X   \\ \hline
\end{tabular}
\end{table}

\bibliographystyle{abnt-alf}
\bibliography{main} % geração automática das referências a partir do arquivo main.bib

\backmatter
\end{document}
