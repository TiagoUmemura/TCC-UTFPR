\chapter{Referencial Teorócio}
Este capítulo apresenta os conceitos de entropia, authorship, ownership, experience e qualidade de software que foram estudados em outras pesquisas.

\section{Entropia de mudança}
A entropia de \citeonline{Shannon:2001:MTC:584091.584093} é uma medida para mensurar a incerteza associada a uma variável que quantifica uma informação contida em uma mensagem produzida por um emissor de dados. É utilizada para determinar a quantidade de bits necessários para identificar unicamente um distribuidor de dados, assim quanto maior a entropia maior a incerteza para identifica-lo. A partir disso foi criado a entropia de mudança com o objetivo de calcular o quanto um código está mudando durante um determinado período de tempo. 

A entropia de mudança introduzida por \citeonline{Hassan:2009:PFU:1555001.1555024} considera que o software é o emissor de dados e as modificações realizadas são os dados a serem considerados. É uma medida para mensurar a quantidade de mudanças que ocorreram em um determinado espaço de tempo em um arquivo de um projeto, as mudança consideradas podem ser obtidas a partir da quantidade de linhas modificadas em um intervalo de tempo ou utilizando número de commits. Hassan também mostrou que alta entropia está relacionada a maior tendência do projeto apresentar falhas.

A pesquisa de \citeonline{Canfora2014} relaciona a entropia de mudança com características do software e atividades de desenvolvimento. As características analisadas foram: refatoração, número de commiters, padrões de projetos e nome de tópicos no projeto. Na pesquisa foram analisados projetos nos sistemas ArgoUML, Eclipse-JDT, Mozilla e Samba em um período de cerca de 10 anos e foi mostrado como esses fatores se relacionam com a entropia, que por consequecia impacta a qualidade do software. Os resultados de Canfora indicam que a entropia de mudança varia conforme o número de desenvolvedores aumentam, sendo que diferentes sistemas de controle de versão apresentam variações diferentes na entropia.

A entropia de mudança é definida como:

\begin{equation}
H(S) = {\sum\limits_{n=1} }\frac{chg(f_i)}{chg(S)}log_2(\frac{chg(f_i)}{chg(S)})
\end{equation}


\section{Authorship e Ownership}
Authorship é uma medida para mensurar o quanto um desenvolvedor contribuiu para um determinado módulo de software e o ownership é o autor com maior authorship, assim um módulo de software pode apresentar ownership forte ou distribuído entre os contribuidores.

Essas duas medidas podem ser obtidas de várias formas: contando número de arquivos que o desenvolvedor modificou, número de commits e outra possibilidade é contar número de linhas modificadas pelo contribuidor, também chamada de code churn\cite{Munson:1998:CCM:850947.853326}.

No trabalho de Greiler e Kim Herzig\cite{Greiler} é feito um estudo para relacionar Ownership com a qualidade do código. Para medir a qualidade do código é considerado o número de bugs que foram corrigidos, número de diretórios e arquivos defeituosos e o Ownership é medido considerando número de contribuidores de um arquivo e também é verificado se existe um contribuidor principal, nesse caso a medida foi calculada com base no número de commits do autor em relação ao total de commits para aquele componente. Também foi mostrado que módulos com ownership fraco tendem a apresentar mais defeitos. Nesse mesmo trabalho foi estudado quatro diferentes software para Windows e é possível observar que nos sistemas analisados a métrica de ownership está correlacionado com número de defeitos. Isso é observado na análise feita tanto em nível de arquivo quanto em nível de diretório.  
Essa pesquisa mostrou que o número de contribuidores e a porcentagem de mudanças aplicadas pelo ownership são bons indicadores de qualidade , mostrando que há relação entre ownership e a qualidade de um projeto apesar da quantidade de erros também depender do tamanho do projeto.

Na pesquisa realizada por \citeonline{Rahman2011} é analisado a relação entre o número de autores com o ownership e com a qualidade do código. O authorship é calculado utilizando o número de linhas modificadas no código pelo desenvolvedor dividido pelo número total de linhas do arquivo, e o autor com a maior contribuição é denomidado ownership. Também é definido implicated code, que é o código modificado quando é corrigido um determinado erro no módulo de software. O trabalho de Rahman investiga a relação entre ownership, authorship e experience com implicated code. Para cada linha de código modificado é utilizado o comando blame para identificar o autor responsável por essa mudança. O resultado fornece evidências indicando que implicated code tende a ser mais frequentemente gerado por poucos autores, vários fragmentos de códigos modificados tem apenas um único autor.

O artigo de \citeonline{Foucault2015} também estuda como o ownership impacta na qualidade do código. Nele os contribuidores são classificados como owner, minor e major. Owner é o contribuidor com maior valor de contribuição, minor o desenvolvedor que contribuiu com menos de 5\% e major contribuiu com mais de 5\%. Após medir as métricas é comparado o número de erros com o a medida de ownership.

Além disso também há pesquisas\cite{Thongtanunam} que analisaram a diferença entre a contribuição de code authoring e reviewing e como a atividade de code reviewing afetam projetos com vários autores que contribuem pouco o que ajuda a compreender melhor o impacto do ownership na qualidade do código.

\section{Experience}
A experience do desenvolvedor influência na produtividade do mesmo e na qualidade do código produzido como foi mostrado em outras pesquisas\cite{Rahman2011}\cite{10.2307/2634607} e quanto mais o desenvolvedor trabalha em diferentes componentes do sistema maior será a sua experience.

Já foi feito pesquisas\cite{Rahman2011} mostrando a relação entre experience e qualidade do código além disso a experience é dividida em dois tipos: a experience especializada e experience geral. A experience especializada é medida considerando o quanto um indivíduo contribui em um determinado arquivo e a experiência geral é medida conseiderando um projeto inteiro.

A experience é a medida para calcular o nível de experiência do contribuidor, essa medida é computada analisando o número de linhas\cite{Rahman2011} deltas comitadas pelo contribuidor em determinado espaço de tempo. Além disso Rahman já mostrou que experiência especializada leva a produzir códigos com menor número de erros e ainda não foi possível concluir que a experiência geral contribui para aumentar a qualidade. No trabalho de Rahman foi correlacionado implicated code com experience.

