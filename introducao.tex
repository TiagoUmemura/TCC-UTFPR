\chapter{Introdução}

Os artefatos de softwares são modificados ao longo do tempo e nesse processo a qualidade do software tende a piorar devido a falta de recursos, tempo e métodos para manter a qualidade do projeto o que aumenta a complexidade do código e dificulta futuras manuntenções.

Para quantificar o impacto das mudanças na qualidade do software os pesquisadores tem utilizado o conceito de entropia de mudança. A entropia é obtida a partir do número de mudanças que ocorrem em um projeto ou arquivo em um determinado período de tempo. Nesse sentido é possível relacionar a quantidade de mudanças com o aumento do valor da entropia, assim, quanto maior o número de mudanças maior será a entropia.

É possível relacionar o aumento da entropia com a degradação da arquitetura do projeto e maior tendência do projeto apresentar falhas.

Como já foi mostrado por \citeonline{Hassan:2009:PFU:1555001.1555024} o aumento da entropia está relacionado com maior tendência do software apresentar problemas e assim diminuir a qualidade. O impacto da entropia na qualidade do projeto também é investigado na pesquisa de \citeonline{Canfora2014}, que analisou a relação entre a entropia e atividades de desenvolvimento, como a refatoração. Em contrapartida estudos mostram o impacto do aspecto social na qualidade do software , por exemplo, \citeonline{Rahman2011} e \citeonline{Foucault2015} analisaram como a qualidade de um projeto é impactado pelo ownership, um fator social. 

É possível observar que foi feito pouco estudo em larga escala sobre como a entropia impacta diferentes tipos de software e também estudos que monitorem os efeitos da entropia em um projeto. Além disso, não foi estudado a relação da entropia com os fatores sociais, por exemplo, como o ownership se relaciona com a entropia. Assim o objetivo desse trabalho é analisar a relação entre entropia, fatores socias e qualidade.

Será investigado quais fatores tem maior impacto na entropia e se maior entropia leva ao débito social.

Este Artigo está organizado da seguinte forma. O capítulo 2 apresenta as definições de cada métrica e o que outros pesquisadores desenvolveram para medir esses fatores e relacionar com a qualidade do código. O capítulo 3 apresenta o método utilizado para medir a entropia, ownership, authorship e experience, motivo dessas métricas serem escolhidas e como será realizada a coleta de dados. O capítulo 4 apresenta os resultados preliminares.