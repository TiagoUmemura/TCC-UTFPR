\chapter{Introdução}

Os artefatos de softwares são modificados ao longo do tempo e nesse processo a qualidade do software tende a piorar, uma vez que existe a necessidade da mudança contínua de requisitos durante a evolução do software \cite{Hassan:2009:PFU:1555001.1555024}.

Para quantificar o impacto das mudanças contínuas os pesquisadores tem utilizado o conceito de entropia de mudança \cite{Hassan:2009:PFU:1555001.1555024}. Esta medida pode ser obtida a partir do número de mudanças que ocorrem em um projeto ou arquivo em um determinado período de tempo. \citeonline{Hassan:2009:PFU:1555001.1555024} observou que maior quantidade de mudanças está relacionada com o aumento do valor da entropia e consequentemente está relacionado com maior tendência do software apresentar defeitos. \citeonline{Canfora2014} analisou a relação entre a entropia e atividades de desenvolvimento, como a refatoração, padrões de projetos e a quantidade de desenvolvedores que mudam um determinado arquivo.

Dada a importância da entropia de mudança, é necessário que desenvolvedores e gerentes possam monitorar os valores de entropia dos arquivos e a possível relação do aumento da entropia com as métricas de software. Não foram encontrados estudos que tenham propostos ferramentas que possibilitem o monitoramento da relação da entropia com as métricas de softwares.

Portanto não existe uma ferramenta que possibilite os desenvolvedores e os gerentes monitorarem os efeitos da entropia e sua relação com métricas sociais, de autoria e de processo.

Diante desse contexto, sabe-se que o desenvolvimento de software é uma tarefa sóciotécnica, pois é essencial a comunicação entre os desenvolvedores para coordenar as atividades de desenvolvimento durante a evolução do software e uma ferramenta poderia monitorar os aspectos sociais e técnicos do projeto para auxiliar os desenvolvedores. Assim o objetivo deste trabalho é construir uma ferramenta que monitora os valores de entropia e a sua relação com as métricas sociais (número de comentários em \textit{Pull Request} e número de \textit{Pull Request}), de autoria de mudança (\textit{authorship}, \textit{ownership} e experiência) e métricas de processo (quantidade de \textit{commits}, quantidade de defeitos, quantidade de linhas removidas, quantidade de linhas adicionadas, \textit{Code Churn}, quantidade de refatorações, \textit{max change set} e \textit{average change set}).

A ferramenta irá gerar relatórios estatísticos sobre a entropia e as métricas e fornecerá a visualização desses dados para auxiliar os desenvolvedores na tomada de decisões durante o desenvolvimento de um projeto. Na visualização de dados será utilizado o \textit{Treemapping} para  mostrar quais arquivos do projeto possuem maior entropia e para as métricas de software serão utilizados gráficos que mostrem os valores das métricas ao longo do tempo.

Para avaliar a ferramenta serão convidados desenvolvedores que responderão um questionário sobre as funcionalidades e usabilidade da ferramenta.  

Esta proposta está organizada da seguinte forma. O capítulo 2 apresenta as definições das métricas sociais, de processos e os trabalhos relacionados. O capítulo 3 apresenta a proposta de construção da ferramenta, como a ferramenta será validada e uso da ferramenta. O capítulo 4 apresenta o cronograma.